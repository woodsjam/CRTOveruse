\documentclass{article}
\author{James Woods}
\title{Who Doesn't Know How Much the Ball Costs?:  Overuse of Common Questions in Choice Experiments}


\usepackage{Sweave}
\begin{document}
\maketitle
\input{DraftWoods-concordance}

\begin{abstract}

Our experimental population, our participants, are a limited resource.  Each time we go into the field with a survey or an experiment and use a common battery of questions, their familiarity with those questions biases responses and can alter our conclusions.  We show that for a common battery of questions, Frederick's three-question Cognitive Reflection Test (CRT),  that participants in several populations, are familiar with the questions and that this bias the correct response rate.  We propose a method for adapting to this phenomena, demonstrate the effects using data from a recent publication and discuss ways that researchers can mitigate effects in the future.

\end{abstract}


\section{Introduction}
  
  \begin{enumerate}
    \item Lots of surprising results in the literature in particular that the CRT is not relelevent in some senses.
    \item evidence that it is being over used
    \item pitch that with messy experiments we have to use more sophisticated statistics
  \end{enumerate}

\section{Use of CRT and Response Trends}
  
  \begin{enumerate}
      \item  Add hoc observations
      \item current trends published
      \item as current research by those guys in the email.
  \end{enumerate}

\section{Statistical Correction for CRT Overstatement}

  \begin{enumerate}
    \item key is to have the count of CRT correct and overstatement, i.e., actual count + a random varible for the deception.
    \item The gamma distribution is the easiest to use.  It is >0 and can be zero, no deception, if need be.  So, it nests the truthfulnes case.
    \item The scale interacts with the paramter on the CRT in the model, so odd things can happen.  
  \end{enumerate}

\section{Comparison of Results}

  \begin{enumerate}
  \item using data from \cite{taylor2013bias}
    \item Simple case in logit or probit modeling.  Gamma is just nested and still identifiable.
    \item maximum likelhood estimate yields.
    \item compare to without gamma.
  \end{enumerate}

  \subsection{Luce Specification}
  
\begin{equation}
\frac{EU_R^{\frac{1}{\mu}}}{EU_L^{\frac{1}{\mu}}-EU_R^{\frac{1}{\mu}}}
\end{equation}
    
    \begin{enumerate}
      \item  No need to wrap the Luce spec in a normal CDF.
      \item Warning the distributions are not well behaved since the difference of two gammas is a tricky little distribution.
      \item Compare the two results.
    \end{enumerate}



  \subsection{Fechner Specification}

  
  
\begin{equation}
\frac{EU_R^{\frac{1}{\mu}}-EU_L^{\frac{1}{\mu}}}{\mu}
\end{equation}


    \begin{enumerate}
      \item  Need to wrap in a normal CDF.
      \item Warning the distributions are not well behaved since the difference of two gammas is a tricky little distribution.
      \item Compare the two results.
    \end{enumerate}


\section{Future Work}


% BibTeX users please use one of
%\bibliographystyle{spbasic}      % basic style, author-year citations
%\bibliographystyle{spmpsci}      % mathematics and physical sciences
%\bibliographystyle{spphys}       % APS-like style for physics
%\bibliography{}   % name your BibTeX data base

% Non-BibTeX users please use

\nocite{*}
\bibliographystyle{plain}
\bibliography{FullBib}



\end{document}
